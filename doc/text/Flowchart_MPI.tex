% -*- root: ../main.tex -*-
\pagestyle{empty}
\begin{tikzpicture}[%
    >=triangle 60,              % Nice arrows; your taste may be different
    start chain=going below,    % General flow is top-to-bottom
    node distance=6mm and 60mm, % Global setup of box spacing
    every join/.style={norm},   % Default linetype for connecting boxes
    ]
% ------------------------------------------------- 
% A few box styles 
% <on chain> *and* <on grid> reduce the need for manual relative
% positioning of nodes
\tikzset{
  base/.style={draw, on chain, on grid, align=center, minimum height=4ex},
  placeholder/.style={base, white, on chain, text width=12em},
  proc/.style={base, rectangle, text width=12em},
  riccati/.style={proc, fill=blue!10},
  test/.style={base, diamond, aspect=2, text width=7em},
  term/.style={proc, rounded corners},
  % coord node style is used for placing corners of connecting lines
  coord/.style={coordinate, on chain, on grid, node distance=6mm and 25mm},
  % nmark node style is used for coordinate debugging marks
  nmark/.style={draw, cyan, circle, font={\sffamily\bfseries}},
  % -------------------------------------------------
  % Connector line styles for different parts of the diagram
  norm/.style={->, draw, lcnorm},
  free/.style={->, draw, lcfree},
  cong/.style={->, draw, lccong},
  dashedline/.style={->, draw, black!50, dashed},
  it/.style={font={\small\itshape}}
}

\node [term] {Löse $N-1$ Anfangswertprobleme des Quadrocopter Modells};
\node [test, join] (test_rank_id) {Prozess Id $=0$?};
\node [proc, join] (sende) {Sende alle Anfangsbedingungen an alle Prozesse};
\node [test, join] (bwork) {sum(bwork) $> 0$?};
\node [proc] (zufall){Wähle zufallsmäßig aus den noch nicht gelösen ODE's aus};
\node [proc, join] {Setze den Status jener ODE im ``bwork'' Array die ausgewählt wurde};
\node [proc, join] {Sende aktuelles ``bwork'' Array an alle Prozesse };
\node [proc, join] {Löse die ODE};
\node [proc, join] (save) {Speichere die Daten und setze ``itmp\_work''};

\node [coord, right=of sende] (sendeleft) {};
\node [coord, left=of save] (saveleft){};
\node [coord, left=of bwork] (bworkleft) {};

\draw [->,lcnorm] (save.west) -- ([xshift=-0.75em]saveleft) -- ([xshift=-0.75em]bworkleft) -- (bwork.west);

\path (test_rank_id.south) to node [near start, xshift=1em] {Ja} (sende);
   \draw [->,lcnorm] (test_rank_id.south) -- (sende);

\path (bwork.south) to node [near start, xshift=1em] {Ja} (zufall);
   \draw [->,lcnorm] (bwork.south) -- (zufall);


%----%
\node [proc, right=of test_rank_id] (wait){Warte auf Daten};
\path (test_rank_id.east) to node [near start, xshift=1em, yshift=1em] {Nein} (wait);
   \draw [->,lcnorm] (test_rank_id.east) -- (wait);

\draw [->,lcnorm] (wait.south) -- ([xshift=9.15em, yshift=-1em]sendeleft) -- (bwork.north);



%---%
\node [test, right=of bwork] (test_rank_id1) {Prozess Id $=0$?};
\node [proc, join] (reciv0){Empfange Lösungen von anderen Prozessen};
\node [test, join] (reciv) {Alle Lösungen empfangen?};
\node [proc] (prepare) {Bereite Daten auf};
\node [term, join] (ende){Prozedurende};

\node [coord, right =of reciv0] (recivright0) {};
\node [coord, right =of reciv] (recivright) {};
\node [coord, right =of ende] (enderight){};


\path (reciv.east) to node [near start, xshift=1em, yshift=1em] {Nein} (recivright);
   \draw [->,lcnorm] (reciv.east) -- ([xshift=1em]recivright) -- ([xshift=1em]recivright0) -- (reciv0);

\path (bwork.east) to node [near start, xshift=1em, yshift=1em] {Nein} (test_rank_id1.west);
   \draw [->,lcnorm] (bwork.east) -- (test_rank_id1.west);

\path (reciv.south) to node [near start, xshift=1em] {Ja} (prepare.north);
   \draw [->,lcnorm] (reciv.south) -- (prepare.north);

%---%
\node [proc, right=of test_rank_id1] (sendsol) {Sende Lösungen und ``itmp\_work''};

\path (test_rank_id1.east) to node [near start, xshift=1em, yshift=1em] {Nein} (sendsol.west);
   \draw [->,lcnorm] (test_rank_id1.east) -- (sendsol.west);
\draw [->,lcnorm] (sendsol.south) -- ([xshift=9.15em]enderight.north) -- (ende);

\end{tikzpicture}